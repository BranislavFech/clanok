\documentclass{beamer}

\usetheme{Antibes}

\usecolortheme{rose}

\setbeamercovered{transparent}

\usepackage[slovak]{babel}
\usepackage[T1]{fontenc}
\usepackage[utf8]{inputenc}
\usepackage{url}

\usepackage{listings}

\lstset{basicstyle=\fontsize{8}{9.6}\selectfont,showstringspaces=false,columns=fullflexible,identifierstyle=\ttfamily,keywordstyle=\bfseries,showstringspaces=false,columns=fullflexible}

\def\BibTeX{\textsc{Bib}\kern-.08em\TeX} 

\newcommand{\footcite}[1]{\footnote{\tiny #1}}
\newcommand{\umlet}{.5}
\newcommand{\emp}[1]{\textit{\alert{#1}}}
\newcommand{\kw}[1]{\mbox{\textbf{#1}}}
\newcommand{\id}[1]{\texttt{#1}}
\newcommand{\stl}{\guillemotleft}
\newcommand{\str}{\guillemotright}

\newcommand{\lsti}{\lstinline[basicstyle=\fontsize{10.5}{12.1}\selectfont]}

\newcommand{\ssection}[1]{
	\section{#1}
	\begin{frame}[fragile=singleslide]\frametitle{}
	\Huge #1
	\end{frame}
}

\newcommand{\ssectionn}[1]{
	\section*{#1}
	\begin{frame}[fragile=singleslide]\frametitle{}
	\Huge #1
	\end{frame}
}

\newenvironment{program}{\begin{beamercolorbox}[rounded=true,shadow=true]{block body}\vspace{-4mm}}{\vspace{-2mm}\end{beamercolorbox}}

\setbeamercolor{fvystup}{fg=white,bg=black}
\newenvironment{vystup}{\begin{beamercolorbox}[rounded=true,shadow=true]{fvystup}}{\end{beamercolorbox}}

\newenvironment{poznamka}{\begin{beamercolorbox}[rounded=true,shadow=false]{block body}}{\end{beamercolorbox}}

\setbeamertemplate{footline}[page number]{}



\author{Branislav Fech}
%\url{www.fiit.stuba.sk/~vranic}, \url{vranic@fiit.stuba.sk}}
%{\tiny \url{www.fiit.stuba.sk/~vranic}, \url{vranic@fiit.stuba.sk}}
\institute{
	Ústav informatiky, informačných systémov a softvérového inžinierstva\\
	Fakulta informatiky a informačných technológií\\
	Slovenská technická univerzita v Bratislave}

\subtitle{\vspace{3mm} Metódy inžinierskej práce 2022/2023}

\title{Jedinečné svetetlné efekty v hrách}

\date{\footnotesize 18. november 2022}




\begin{document}

\begin{frame}[fragile=singleslide]
\titlepage
\end{frame}


\begin{frame}[fragile=singleslide]\frametitle{O čom to je}

\begin{itemize}
\item Spracovanie svetla hraje v hrách podstatnú rolu
\item Nie je možné pre počítač obyčajného hráča vypočítať každú časticu zvlášť
\item Je potrebné nájsť a porovnať rôzne spôsoby spracovania a vyobrazenia svetelných efektov
\end{itemize}
	
\end{frame}


\begin{frame}[fragile=singleslide]\frametitle{Prehľad}
\tableofcontents
\end{frame}


\section{Svetelné efekty}

\begin{frame}[fragile=singleslide]\frametitle{Kaustika}
\begin{itemize}
\item Pozorovateľné na dne bazéna
\item Nadmerne osvetlená plocha
	\begin{enumerate}
	\item Svetelné lúče prechádzajú médiom
	\item Lámu sa
	\item Zbiehajú sa v jednom mieste
	\end{enumerate}
\item Pozorovateľné aj pri sklenených objektoch
\end{itemize}
\end{frame}

\begin{frame}[fragile=singleslide]\frametitle{Príklad kaustiky}
\begin{center}
\includegraphics[scale=.20]{Kaustika.jpg}

{\tiny Kaustika na vodnom dne\ldots}
\end{center}
\end{frame}

\begin{frame}[fragile=singleslide]\frametitle{Svetelné lúče}
\begin{itemize}
\item Kontrastný lúč svetla v zatienenom prostredí
	\begin{itemize}
	\item Viditeľné vďaka časticiam prachu vo vzduchu
	\end{itemize}
\item Časť svetla je blokovaná objektom 
\item Neblokované svetlo vidíme v podobe svetelných lúčov
\end{itemize}
\begin{center}
\includegraphics[scale=.5]{Svetelne_luce.jpg}
\end{center}
\end{frame}

\begin{frame}[fragile=singleslide]\frametitle{Krepuskulárne lúče}
\begin{itemize}
\item Špecifický prípad svetelných lúčov
\item Časť svetla je blokovaná priesvitnými objektami
	\begin{itemize}
	\item Oblaky
	\item Voda
	\end{itemize}
\item Rozdielne intenzity svetelných lúčov
\end{itemize}
\begin{center}
\includegraphics[scale=.15]{god_ray.jpg}
\end{center}
\end{frame}

\section{Spracovanie}

\begin{frame}[fragile=singleslide]\frametitle{Post-processing}
\begin{itemize}
\item Krepuskulárne lúče sú len pridané do obrázka využitím jednoduchých efektov
\item Rýchle vykresľovanie
\item Funkčné len ak je zdroj svetla v obraze
\end{itemize}
\end{frame}

\begin{frame}[fragile=singleslide]\frametitle{Mapovanie tieňov}
\begin{itemize}
\item Detekcia oklúzie vo svetelných lúčoch
\item Funkčný len v homogénnom prostredí, ale
	\begin{itemize}
	\item Schopný pracovať s dynamickým zdrojom svetla
	\item Schopný pracovať s dynamickým predmetom pôsobiacim ako oklúzia
	\end{itemize}
\item Náročný na výpočet
\begin{itemize}
	\item Preložené vzorkovanie na zníženie výpočtovej doby
	\end{itemize}
\end{itemize}
\end{frame}

\section{Porovnanie}

\begin{frame}[fragile=singleslide]\frametitle{Výhody a nevýhody oboch metód}
\begin{center}
\includegraphics[scale=.65]{tabulla.png}
\end{center}
\end{frame}


\section*{Zhodnotenie}

\begin{frame}[fragile=singleslide]\frametitle{Zhodnotenie}
\begin{itemize}
\item Je potrebné rozhodnúť sa medzi\ldots{}
    \begin{itemize}
    \item Pomalším ale prirodzenejším riešením
    \item Rýchlejším ale menej uspokojivým riešením
    \end{itemize}
\item Ale je aj v dnešnej dobe veľa miesta pre rozvoj\ldots{}
\end{itemize}
\end{frame}


\end{document}

Text \end{document} za príkazom \end{document}