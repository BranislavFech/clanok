% Metódy inžinierskej práce

\documentclass[10pt,twoside,slovak,a4paper]{article}

\usepackage[slovak]{babel}
%\usepackage[T1]{fontenc}
\usepackage[IL2]{fontenc} % lepšia sadzba písmena Ľ než v T1
\usepackage[utf8]{inputenc}
\usepackage{graphicx}
\usepackage{url} % príkaz \url na formátovanie URL
\usepackage{hyperref} % odkazy v texte budú aktívne (pri niektorých triedach dokumentov spôsobuje posun textu)

\usepackage{cite}
%\usepackage{times}

\pagestyle{headings}

\title{Jedinečné svetlené efekty v hrách\thanks{Semestrálny projekt v predmete Metódy inžinierskej práce, ak. rok 2022/23, vedenie: Igor Stupavský}} % meno a priezvisko vyučujúceho na cvičeniach

\author{Branislav Fech\\[2pt]
	{\small Slovenská technická univerzita v Bratislave}\\
	{\small Fakulta informatiky a informačných technológií}\\
	{\small \texttt{xfech@stuba.sk}}
	}

\date{\small 28. september 2022} % upravte



\begin{document}

\maketitle

\begin{abstract}
Prvotriedne spracovanie istých svetelných efektov je kritické v modernom hraní. 
Jeho nedôveryhodnosť je schopné nevedome zruinovať zážitok z hry. Väčšina z nás 
si to ani nemusí uvedomiť, môžeme mať len pocit, že niečo nesedí a často krát 
nehodnovernosť svetelného spracovania ničí pocit realizmu danej hry. Avšak ani 
moderná technológia prístupná obyčajnému človeku nie je schopná spracovať každú 
časticu zvlášť, teda aspoň nie v zlomku sekundy. Preto je potrebné spracovať tieto 
efekty čo najdôveryhodnejšie, avšak zároveň s čo najmenšou záťažou na naše hardware-ové 
vybavenie.
\ldots
\end{abstract}



\section{Úvod}
Málokto by nesúhlasil s tvrdením, že svetlo je jednou z, ak nie najdôležitejšou zložkou 
skvelo graficky spracovanej hry. Narozdiel od nedokončených textúr, alebo nedostatku 
detialov, svetlo možno nespozorujeme na prvý pohľad. Môžeme si len povedať, že je niečo 
inak, ako by malo byť a už to robí hrou menej hodnotnou. V tejto práci sa budem zameriavať 
na zložitejšie prípady svetelných efektov a spôsoby ich vyobrazenia.

\section{Svetelné efekty} \label{se}
Svetlo sa neustále odráža alebo láme od všetkých hmotných vecí. Väčšina svetelných efektov 
pre nás nie je ničím zaujímavá, keďže sa s nimi stretávame každý deň. Sú ale aj také, pri 
ktorch si môžeme povedať, že sme sa ocitli v správny čas na správnom mieste. Tieto efekty 
dodávajú  pozorvateľovi uchvacujúci a jedinečný pocit, teda robia daný moment viac pamätania 
hodný.

\subsection{Kaustika} \label{se:kaustika}
Kaustika je jedným zo svetelných efektov, ktoré môžeme pozorovať napríklad keď svetlo 
prechádza pohárom, alebo na dne bazéna za predpokladu, že voda je číra. Vzníká, keď sa 
svetlo odráža alebo sa láme pri prechode médiom a tieto lúče sa zbiehajú v jednom mieste, 
teda daná plocha je viac osvetlená. Dá sa označiť za jednoduchšiu verziu toho, čo bude 
nasledovať, teda krespuskulárnych lúčov. V niektorých literatúrach sa tieto lúče označujú 
aj ako "3D kaustika."

\subsection{Svetelné lúče} \label{se:luce}
Svetelné lúče je možné pozorovať v scenérii, kedy je väčšina svetla blokovaná istým 
objektom, napríklad stenou. Avšak časť svetla, ktorá nie je blokovaná, napríklad svetlo 
prechádzajúce oknom, alebo deravou strechou, je viditeľná v podobe jedného alebo viacerých 
svetlných lúčov. To vytvára kontrast medzi tmavou časťou scenérie a jedným alebo viacrými 
svetlnými lúčami. Avšak pre výskyt týchto lúčov je potrebné, aby sa vo vzduchu nachádzali 
častice prachu alebo ine mikročastice, keďže od nich sa prechádzajúce svetlo odráža a 
len vďaka nim sú tieto lúče viditeľné.

\subsubsection{Krepuskulárne lúče} \label{se:luce:kl}
Krepuskulárne lúče sú špecifickým prípadom výskytu svetelných lúčov. V tomto prípade svetlo 
nie je blokované pevným objektom, ale oblakmi. Čo robí tento prípad svetelných lúčov 
zaujímavým a pre výpočtovú grafiku náročným je skutočnosť, že oblaky nie sú nepriehľadné, 
ale priesvitné. Teda svetlo je pohlcované v iných množstvách, niekde viac a niekde menej, 
takže svetelné lúče sa vyskytujú v rôznych intenzitách. Narozdiel od svetla blokovaného 
budovou, kde sa môžu vyskytnúť len 2 prípady- buď svetlo prechádza, 
alebo nie.\cite{Crepuscular_rays}

Tieto lúče sa môžu ojedinele vyskytovať aj vo vode v prípade, ak sme na vodnom dne a 
pozeráme sa smerom hore. V taktejto situácii pôsobí voda ako médium a zároveň ako 
rozptyľovač svetelných lúčov. 

\begin{figure}[h]
    \centering
    \includegraphics[scale=0.2]{god_ray.jpg}
    \caption{Krepuskulárne lúče}
    \label{fig:kl}
\end{figure}

\section{Spracovanie} \label{spracovanie}
V prípade spracovania krepuskulárnych lúčov na obyčajnej fotografii nie je zložitosť 
nejako vysoká, nakoľko výsledok nemusí byť hotový v zlomku sekundy. Avšak hry sú niečo 
iné. Úlohou je vypracovať daný obraz čo najhodnovernejšie a zároveň s čo najväčším 
realizmom. Musí to vyzerať hodnoverne, v opačnom prípade skončíme s hrou, ktorú už 
kvôli jej neprirodzenému spracovaniu týchto lúčov už nezapneme.

\subsection{Post-processing} \label{spracovanie:pp}
Táto metóda je bohužiaľ funkčná len v prípade, že zdroj svetla sa nachádza v obraze, 
teda zlyhá, ak zdroj nie je viditeľný. Na druhej strane, výhoda tejto metódy spočíva 
v tom, že úprava obrázka nijako neovplyvňuje čas vykresľovania obrázka, nakoľko sú 
krepuskulárne lúče jednoduchými "efektami" len pridané do obrázka. Táto metóda spočíva 
v rozmazaní obrázka horizontálne aj vertikálne a následným zosvetlením na simuláciu 
lúčov.\cite{Light_shafts}.

\subsection{Mapovanie tieňov} \label{spracovanie:mt}
V roku 2009 bola prednesená ďalšia metóda na spracovanie krepuskulárnych lúčov v 
reálnom čase pomocou mapovania tieňov. Tie slúžia na detekciu oklúzie alebo "prekážky" 
vo svetelných lúčoch. Tento algoritmus je síce schopný fungovať len v homogénnom prostredí, 
ale na druhej strane je schopné pracovať aj s dynamickým zdrojom svetla a taktiež dynamickými 
predmetami pôsobiacimi ako oklúzie. Nakoľko je tento algoritmus náročný na výpočet, je v ňom 
implementované preložené vzorkovanie na zníženie výpočtovej doby\cite{God_rays}.

\section{Porovnanie} 
\label{por}
Obe metódy spomenuté vyššie majú svoje výhody  a nevýhody.

Post processing nám prináša rýchlejšie zobrazenie, keďže úprava sa deje až po tom, čo je 
obraz vykreslený. Nevýhodou však je, že výsledky nemusia byť vždy úplné dôveryhodné a na to, 
aby tento algoritmus fungoval, musí byť zdroj svetla v obraze. 

V prípade použitia mapovania tieňov je nevýhodou pomalšie zobrazenie kvôli nutnosti 
mapovania každého svetlného lúča zvlášť. Výhody prinášané týmto algoritmom sú oveľa 
prirodzenejšie výsledky spolu s tým, že je schopný pracovať s pohyblivým svetelným 
zdrojom aj keď sa nenachádza v zornom poli hráča, narodziel od post-processingu.

\section{Zhrnutie} 
\label{zhr}
Dilemou je rozhodnúť sa medzi pomalším, ale prirodzenejším, alebo rýchlejším, ale 
menej uspokojivým výsledkom. Je logické držať sa blízko fyzicky korektného zobrazenia 
týchto svetelných efektov pre čo najlepšie výsledky. No vykresľovanie týchto scenérií 
v reálnom čase je aj v dnešnej dobe v stave, kedy je veľa miesta pre ďalší rozvoj.


\section{Diskusia} 
\label{dsk}
\paragraph{Udržateľnosť a etika.} Rozvoj je niečo, čo sa nikdy nedalo a ani sa nebude dať 
zastaviť. Mať ale znalosť o možnostiach a rôznych alternatívach, ako zabezpečiť, aby rozvoj 
bol aj udržateľný, bolo by neetické túto znalosť nevyužiť. Pre jednotlivca to nie je nič 
špeciálne, ale ak by mala byť táto znalosť rozšírená medzi masy ľudí, začala by sa 
vykresľovať významná štatistika. Je to spoločenskou aj etickou povinnosťou, každého 
súčasného človeka, schopného sa voľne rozhodovať konať v záujme pre väčšie blaho ďalších 
generácií.

\paragraph{Historické súvislosti.} Je fascinujúce uvedomiť si, kde a ako jedna z najväčších 
priemyselných oblastí vznikla. Teda pri obyčajnom skúmaní toho, ako rôzne  materiály reagujú 
na elektrinu a následne vedieť, k čomu všetkému toto skúmanie viedlo- vzniku tranzistorov, 
logických obvodov, pamätí. Následovné desaťročia plné vylepšení, zmenšovaní obvodov, 
zrychľovaní a ich zvyšujúcej sa zložitosti. V tejto oblasti histórie je ukázané, k čomu môže 
viesť veľa rokov aj malých krokov, malých úspechov, ale aj veľa rokov neúspechov, investícií 
a trpezlivosti.

\paragraph{Technológia a ľudia.} Sú ľudia, pre ktorých je práca v tímoch prirodzenejšia, 
jednoduchšia, teda lepšia. Avšak bez ohľadu na to, akí sme, je potrebné z času na čas 
pracovať nie ako jednotlivec, ale ako skupina. Takáto skupinová práca môže eliminovať nejaké 
alebo dokonca všetky naše ťažkosti pri práci na projekte, nech už sú akékoľvek. Týmto sme 
schopní dať našim spolupracovníkom na jednej strane slobodu pracovať na istej časti daného 
projektu, na druhej strane povinnosť pracovať na danom projekte a pomôcť im s prípadným 
problémom prokrasstinácie. Tímovou prácou sme schopní koordinovať, pozdvihovať a dosahovať 
lepšie výsledky ako samostatnou prácou.



%\acknowledgement{Ak niekomu chcete poďakovať\ldots}


% týmto sa generuje zoznam literatúry z obsahu súboru literatura.bib podľa toho, na čo sa v článku odkazujete
\bibliography{literatura}
\bibliographystyle{abbrv} % prípadne alpha, abbrv alebo hociktorý iný

\begin{figure}[p]
    \centering
    \includegraphics[scale=0.8]{Kresba1.png}
    \caption{Diagram}
    \label{dia}
\end{figure}

\end{document}