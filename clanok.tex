% Metódy inžinierskej práce

\documentclass[10pt,twoside,slovak,a4paper]{article}

\usepackage[slovak]{babel}
%\usepackage[T1]{fontenc}
\usepackage[IL2]{fontenc} % lepšia sadzba písmena Ľ než v T1
\usepackage[utf8]{inputenc}
\usepackage{graphicx}
\usepackage{url} % príkaz \url na formátovanie URL
\usepackage{hyperref} % odkazy v texte budú aktívne (pri niektorých triedach dokumentov spôsobuje posun textu)

\usepackage{cite}
%\usepackage{times}

\pagestyle{headings}

\title{Jedinečné svetlené efekty v hrách\thanks{Semestrálny projekt v predmete Metódy inžinierskej práce, ak. rok 2022/23, vedenie: Igor Stupavský}} % meno a priezvisko vyučujúceho na cvičeniach

\author{Branislav Fech\\[2pt]
	{\small Slovenská technická univerzita v Bratislave}\\
	{\small Fakulta informatiky a informačných technológií}\\
	{\small \texttt{xfech@stuba.sk}}
	}

\date{\small 28. september 2022} % upravte



\begin{document}

\maketitle

\begin{abstract}
Prvotriedne spracovanie istých svetelných efektov je kritické v modernom hraní. 
Jeho nedôveryhodnosť je schopné nevedome zruinovať zážitok z hry. Väčšina z nás 
si to ani nemusí uvedomiť, môžeme mať len pocit, že niečo nesedí a často krát 
nehodnovernosť svetelného spracovania ničí pocit realizmu danej hry. Avšak ani 
moderná technológia prístupná obyčajnému človeku nie je schopná spracovať každú 
časticu zvlášť, teda aspoň nie v zlomku sekundy. Preto je potrebné spracovať tieto 
efekty čo najdôveryhodnejšie, avšak zároveň s čo najmenšou záťažou na naše hardware-ové 
vybavenie.
\ldots
\end{abstract}



\section{Úvod}
Málokto by nesúhlasil s tvrdením, že svetlo je jednou z, ak nie najdôležitejšou 
zložkou skvelo graficky spracovanej hry. Narozdiel od nedokončených textúr, alebo 
nedostatku detialov, svetlo možno nespozorujeme na prvý pohľad. Môžeme si len povedať, 
že je niečo inak, ako by malo byť a už to robí hrou menej hodnotnou. V tejto práci sa 
budem zameriavať na zložitejšie prípady svetelných efektov a spôsoby ich vyobrazenia.

\section{Svetelné efekty} \label{se}
Svetlo sa neustále odráža alebo láme od všetkých hmotných vecí. Väčšina svetelných 
efektov pre nás nie je ničím zaujímavá, keďže sa s nimi stretávame každý deň. Sú ale 
aj také, pri ktorch si môžeme povedať, že sme sa ocitli v správny čas na správnom mieste.
Tieto efekty dodávajú  pozorvateľovi uchvacujúci a jedinečný pocit, teda robia daný 
moment viac pamätania hodný.

\subsection{Kaustika} \label{se:kaustika}
Kaustika je jedným zo svetelných efektov, ktoré môžeme pozorovať napríklad keď svetlo 
prechádza pohárom, alebo na dne bazéna za predpokladu, že voda je číra. Vzníká, keď sa 
svetlo odráža alebo sa láme pri prechode médiom a tieto lúče sa zbiehajú v jednom mieste, 
teda daná plocha je viac osvetlená.

\subsection{Svetelné lúče} \label{se:luce}
Svetelné lúče je možné pozorovať v scenérii, kedy je väčšina svetla blokovaná istým 
objektom, napríklad stenou. Avšak časť svetla, ktorá nie je blokovaná, napríklad svetlo 
prechádzajúce oknom, alebo deravou strechou, je viditeľná v podobne jedného alebo viacerých 
svetlných lúčov. To vytvára kontrast medzi tmavou časťou scenérie a jedným alebo viacrými 
svetlnými lúčami.


%\acknowledgement{Ak niekomu chcete poďakovať\ldots}


% týmto sa generuje zoznam literatúry z obsahu súboru literatura.bib podľa toho, na čo sa v článku odkazujete
\bibliography{literatura}
\bibliographystyle{plain} % prípadne alpha, abbrv alebo hociktorý iný
\end{document}
